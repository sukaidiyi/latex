%!TEX TS-program = xelatex
%!TEX encoding = UTF-8 Unicode

\documentclass[12pt]{article} %这个我就不多说了,头文件
\usepackage{url} %这个我也不多说了
\usepackage{fontspec,xltxtra,xunicode} %最新的mactex都有
\usepackage{amsmath}
\usepackage{hyperref}
\usepackage{setspace}

\hypersetup{
    colorlinks=true,
    linkcolor=blue,
    filecolor=blue,      
    urlcolor=blue,
    citecolor=cyan,
}
\defaultfontfeatures{Mapping=tex-text}
\setromanfont{Songti SC} %设置中文字体
\XeTeXlinebreaklocale “zh”
\XeTeXlinebreakskip = 0pt plus 1pt minus 0.1pt %文章内中文自动换行,可以自行调节
\setlength{\parskip}{1em}

\begin{document}
\begin{enumerate}
\item 极限定义 
    \par $\displaystyle\lim _{x \rightarrow x_{0}} f(x)=A \Leftrightarrow f_{-}\left(x_{0}\right)=f_{+}\left(x_{0}\right)=A$
    \par \href{https://www.matongxue.com/matex/2576/react/}{题目1}
\item 无穷小定义
    \par $\displaystyle\lim _{x \rightarrow x_{0}} f(x)=A \Leftrightarrow f\left(x\right)=A+\alpha$
    其中$\displaystyle\lim_{x\to x_0}\alpha=0$
    \par \href{https://www.matongxue.com/matex/2575/react/}{题目1} 
\item 保号定理
     \par 设$\displaystyle\lim_{x\to x_0}f(x)=A$,又$A > 0$或$A < 0$,则$\exists\delta > 0$
     \par 当$x\in(x_0-\delta,x_0+\delta)$,且$x\neq x_0$时,$f(x)>0$(或$f(x)<0$)
     \par \href{https://www.matongxue.com/matex/2578/react/}{题目1} 
\item 无穷小比阶
    设$\lim\alpha(x)=0,\lim\beta(x)=0,\beta\neq 0$
    \begin{enumerate}
        \item 高阶无穷小
        \par 若$\displaystyle\frac{\alpha(x)}{\beta(x)}=0$,则$\alpha(x)$是$\beta(x)$高阶的无穷小,
        \par 记为$\alpha(x)=o(\beta(x))$ \quad 
        \par \href{https://www.matongxue.com/matex/2579/react/}{题目1} 
        \item 低阶无穷小
        \par 若$\displaystyle\frac{\alpha(x)}{\beta(x)}=\infty$,则$\alpha(x)$是$\beta(x)$低阶的无穷小
        \par \href{https://www.matongxue.com/matex/2582/react/}{题目1} 
        \item 同阶无穷小
        \par 若$\displaystyle\frac{\alpha(x)}{\beta(x)}=c(c\neq 0)$,则$\alpha(x)$是$\beta(x)$是同阶的无穷小
        \par \href{https://www.matongxue.com/matex/2583/react/}{题目1} 
        \item k阶无穷小
        \par 若$\displaystyle\frac{\alpha(x)}{\beta^k(x)}=c(c\neq 0,k>0)$,则$\alpha(x)$是$\beta(x)$是k阶的无穷小
        \par \href{https://www.matongxue.com/matex/2581/react/}{题目1} 
        \item 等价无穷小
        \par 若$\displaystyle\frac{\alpha(x)}{\beta(x)}=1$,则$\alpha(x)$是$\beta(x)$是等价的无穷小
        \par 记为$\alpha(x)\sim\beta(x)$
        \par \href{https://www.matongxue.com/matex/2580/react/}{题目1} 
    \end{enumerate}
\item 常用的等价无穷小
    \begin{enumerate}
        \item $\sin x\sim x$\quad \href{https://www.matongxue.com/matex/2584/react/}{题目1} 
        \item $\arcsin x\sim x$\quad \href{https://www.matongxue.com/matex/2585/react/}{题目1} 
        \item $\tan x\sim x$\quad \href{https://www.matongxue.com/matex/2586/react/}{题目1} 
        \item $\arctan x\sim x$\quad \href{https://www.matongxue.com/matex/2587/react/}{题目1} 
        \item $\ln(1+x)\sim x$\quad \href{https://www.matongxue.com/matex/2588/react/}{题目1} 
        \item $\displaystyle\ln(1+x)\sim \frac{x}{\ln a}$\quad \href{https://www.matongxue.com/matex/2590/react/}{题目1} 
        \item $e^x-1 \sim x$\quad \href{https://www.matongxue.com/matex/2589/react/}{题目1} 
        \item $1-\cos x\sim \frac{1}{2}x^2$\quad \href{https://www.matongxue.com/matex/2591/react/}{题目1} 
        \item $(1+x)^{\frac{1}{n}}-1\sim \frac{1}{n}x$\quad \href{https://www.matongxue.com/matex/2592/react/}{题目1} 
    \end{enumerate}
\item 无穷小的性质
    \begin{enumerate}
        \item 有限个无穷小的代数和为无穷小\quad \href{https://www.matongxue.com/matex/2593/react/}{题目1} 
        \item 有限个无穷小的乘积为无穷小\quad \href{https://www.matongxue.com/matex/2594/react/}{题目1} 
        \item 无穷小乘以有界变量为无穷小\quad \href{https://www.matongxue.com/matex/2595/react/}{题目1} 
    \end{enumerate}
\item 在同一变化趋势下
    \begin{enumerate}
        \item 无穷大的倒数为无穷小\quad \href{https://www.matongxue.com/matex/2596/react/}{题目1}
        \item 无穷小的倒数为无穷大\quad \href{https://www.matongxue.com/matex/2597/react/}{题目1}
    \end{enumerate}
\item 极限四则运算
    $\lim f(x)=A,\lim g(x)=B$,则
    \begin{enumerate}
        \item $\lim (f(x) \pm g(x))=A \pm B$\quad \href{https://www.matongxue.com/matex/2598/react/}{题目1}
        \item $\lim f(x)g(x)=AB$\quad \href{https://www.matongxue.com/matex/2599/react/}{题目1}
        \item $\displaystyle\lim \frac{f(x)}{g(x)}=\frac{A}{B}(B \neq 0)$
        \quad \href{https://www.matongxue.com/matex/2600/react/}{题目1}
    \end{enumerate}
\item 夹逼定理
    \begin{enumerate} 
        \item 数列版
        \par 如果数列$\{x_n\},\{y_n\},\{z_n\}$满足$y_n\leq x_n \leq z_n(n=1,2,3,\cdots)$
        \par 且$\displaystyle\lim_{n\to\infty}y_n=A,\lim_{n\to\infty x_n}=A$,则$\displaystyle\lim_{n\to\infty}x_n=A$
        \par \href{https://www.matongxue.com/matex/2602/react/}{题目1}
        \item 函数版
        \par 设在$x_0$的邻域内,恒有$\varphi(x)\leq f(x) \leq \phi(x)$
        \par 且$\displaystyle\lim_{x\to x_0}\varphi(x)=\lim_{x\to x_0}\phi(x)=A$,
        则$\displaystyle\lim_{x\to x_0}f(x)=A$
        \par \href{https://www.matongxue.com/matex/2601/react/}{题目1}
    \end{enumerate}
\item 单调有界定理:单调有界的数列必有极限
    \par \href{https://www.matongxue.com/matex/2603/react/}{题目1}
\item 两个重要的极限
    \begin{enumerate}
        \item $\displaystyle\lim_{x\to 0}\frac{\sin x}{x}=1$
        \quad \href{https://www.matongxue.com/matex/2604/react/}{题目1}
        \item $\displaystyle\lim_{x\to 0}(1+x)^{\frac{1}{x}}=e$
        \quad \href{https://www.matongxue.com/matex/2605/react/}{题目1}
    \end{enumerate}
\item 有理分式极限
$$
\lim _{x \rightarrow \infty} \frac{a_{0} x^{n}+a_{1} x^{n-1}+\cdots+a_{n-1} x+a_{n}}{b_{0} x^{m}+b_{1} x^{m-1}+\cdots+b_{m-1} x+b_{m}}=\left\{\begin{array}{l}{
    \frac{a_{0}}{b_{0}}, n=m} \\ 
    {0, n<m} \\ 
{\infty, n>m}\end{array}\right.
$$
\par \href{https://www.matongxue.com/matex/2606/react/}{题目1}
\item 常用的极限
    \begin{enumerate}
        \item $\displaystyle\lim_{n\to\infty}\sqrt[n]{n}=1$
        \quad \href{https://www.matongxue.com/matex/2608/react/}{题目1}
        \item $\displaystyle\lim_{x\to+\infty}\arctan x=\frac{\pi}{2}$
        \quad \href{https://www.matongxue.com/matex/2609/react/}{题目1}
        \item $\displaystyle\lim_{x\to-\infty}\arctan x=-\frac{\pi}{2}$
        \quad \href{https://www.matongxue.com/matex/2610/react/}{题目1}
        \item $\displaystyle\lim_{x\to+\infty}arccot x=0$
        \quad \href{https://www.matongxue.com/matex/2611/react/}{题目1}
        \item $\displaystyle\lim_{x\to-\infty}arccot x=\pi$
        \quad \href{https://www.matongxue.com/matex/2612/react/}{题目1}
        \item $\displaystyle\lim_{x\to-\infty}e^x=0$
        \quad \href{https://www.matongxue.com/matex/2613/react/}{题目1}
        \item $\displaystyle\lim_{x\to+\infty}e^x=+\infty$
        \quad \href{https://www.matongxue.com/matex/2614/react/}{题目1}
        \item $\displaystyle\lim_{x\to 0^{+}}x^x=1$
        \quad \href{https://www.matongxue.com/matex/2615/react/}{题目1}
    \end{enumerate}
\end{enumerate}
\end{document}